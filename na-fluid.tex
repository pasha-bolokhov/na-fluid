\documentclass[epsfig,12pt]{article}
\usepackage{epsfig}
\usepackage{graphicx}
\usepackage{rotating}
\usepackage{latexsym}
\usepackage{amsmath}
\usepackage{amssymb}
\usepackage{relsize}
\usepackage{geometry}
\geometry{letterpaper}
\usepackage{color}
\usepackage{bm}
\usepackage{slashed}
\usepackage{showlabels}




%%%%%%%%%%%%%%%%%%%%%%%%%%%%%%%%%%%%%%%%%%%%%%%%%%%%%%%%%%%%%%%%%%%%%%%%%%%%%%%%
%                                                                              %
%                                                                              %
%                     D O C U M E N T   S E T T I N G S                        %
%                                                                              %
%                                                                              %
%%%%%%%%%%%%%%%%%%%%%%%%%%%%%%%%%%%%%%%%%%%%%%%%%%%%%%%%%%%%%%%%%%%%%%%%%%%%%%%%
\def\baselinestretch{1.1}
\renewcommand{\theequation}{\thesection.\arabic{equation}}

\hyphenation{con-fi-ning}
\hyphenation{Cou-lomb}
\hyphenation{Yan-ki-e-lo-wicz}




%%%%%%%%%%%%%%%%%%%%%%%%%%%%%%%%%%%%%%%%%%%%%%%%%%%%%%%%%%%%%%%%%%%%%%%%%%%%%%%%
%                                                                              %
%                                                                              %
%                      C O M M O N   D E F I N I T I O N S                     %
%                                                                              %
%                                                                              %
%%%%%%%%%%%%%%%%%%%%%%%%%%%%%%%%%%%%%%%%%%%%%%%%%%%%%%%%%%%%%%%%%%%%%%%%%%%%%%%%
\def\beq{\begin{equation}}
\def\eeq{\end{equation}}
\def\beqn{\begin{eqnarray}}
\def\eeqn{\end{eqnarray}}
\def\beqn{\begin{eqnarray}}
\def\eeqn{\end{eqnarray}}
\def\nn{\nonumber}
\def\ba{\beq\new\begin{array}{c}}
\def\ea{\end{array}\eeq}
\def\be{\ba}
\def\ee{\ea}


\newcommand{\nfour}{${\cal N}=4\;$}
\newcommand{\none}{${\mathcal N}=1\,$}
\newcommand{\nonen}{${\mathcal N}=1$}
\newcommand{\ntwo}{${\mathcal N}=2$}
\newcommand{\ntt}{${\mathcal N}=(2,2)\,$}
\newcommand{\nzt}{${\mathcal N}=(0,2)\,$}
\newcommand{\ntwon}{${\mathcal N}=2$}
\newcommand{\ntwot}{${\mathcal N}= \left(2,2\right) $ }
\newcommand{\ntwoo}{${\mathcal N}= \left(0,2\right) $ }
\newcommand{\ntwoon}{${\mathcal N}= \left(0,2\right)$}


\newcommand{\ca}{{\mathcal A}}
\newcommand{\cell}{{\mathcal L}}
\newcommand{\cw}{{\mathcal W}}
\newcommand{\cs}{{\mathcal S}}
\newcommand{\vp}{\varphi}
\newcommand{\pt}{\partial}
\newcommand{\ve}{\varepsilon}
\newcommand{\gs}{g^{2}}
\newcommand{\zn}{$Z_N$}
\newcommand{\cd}{${\mathcal D}$}
\newcommand{\cde}{{\mathcal D}}
\newcommand{\cf}{${\mathcal F}$}
\newcommand{\cfe}{{\mathcal F}}
\newcommand{\ff}{\mc{F}}
\newcommand{\bff}{\ov{\mc{F}}}


\newcommand{\p}{\partial}
\newcommand{\wt}{\widetilde}
\newcommand{\ov}{\overline}
\newcommand{\mc}[1]{\mathcal{#1}}
\newcommand{\md}{\mathcal{D}}
\newcommand{\ml}{\mathcal{L}}
\newcommand{\mw}{\mathcal{W}}
\newcommand{\ma}{\mathcal{A}}


\newcommand{\GeV}{{\rm GeV}}
\newcommand{\eV}{{\rm eV}}
\newcommand{\Heff}{{\mathcal{H}_{\rm eff}}}
\newcommand{\Leff}{{\mathcal{L}_{\rm eff}}}
\newcommand{\el}{{\rm EM}}
\newcommand{\uflavor}{\mathbf{1}_{\rm flavor}}
\newcommand{\lgr}{\left\lgroup}
\newcommand{\rgr}{\right\rgroup}


\newcommand{\Mpl}{M_{\rm Pl}}
\newcommand{\suc}{{{\rm SU}_{\rm C}(3)}}
\newcommand{\sul}{{{\rm SU}_{\rm L}(2)}}
\newcommand{\sutw}{{\rm SU}(2)}
\newcommand{\suth}{{\rm SU}(3)}
\newcommand{\ue}{{\rm U}(1)}


\newcommand{\LN}{\Lambda_\text{SU($N$)}}
\newcommand{\sunu}{{\rm SU($N$) $\times$ U(1) }}
\newcommand{\sunun}{{\rm SU($N$) $\times$ U(1)}}
\def\cfl {$\text{SU($N$)}_{\rm C+F}$ }
\def\cfln {$\text{SU($N$)}_{\rm C+F}$}
\newcommand{\mUp}{m_{\rm U(1)}^{+}}
\newcommand{\mUm}{m_{\rm U(1)}^{-}}
\newcommand{\mNp}{m_\text{SU($N$)}^{+}}
\newcommand{\mNm}{m_\text{SU($N$)}^{-}}
\newcommand{\AU}{\mc{A}^{\rm U(1)}}
\newcommand{\AN}{\mc{A}^\text{SU($N$)}}
\newcommand{\aU}{a^{\rm U(1)}}
\newcommand{\aN}{a^\text{SU($N$)}}
\newcommand{\baU}{\ov{a}{}^{\rm U(1)}}
\newcommand{\baN}{\ov{a}{}^\text{SU($N$)}}
\newcommand{\lU}{\lambda^{\rm U(1)}}
\newcommand{\lN}{\lambda^\text{SU($N$)}}
\newcommand{\bxir}{\ov{\xi}{}_R}
\newcommand{\bxil}{\ov{\xi}{}_L}
\newcommand{\xir}{\xi_R}
\newcommand{\xil}{\xi_L}
\newcommand{\bzl}{\ov{\zeta}{}_L}
\newcommand{\bzr}{\ov{\zeta}{}_R}
\newcommand{\zr}{\zeta_R}
\newcommand{\zl}{\zeta_L}
\newcommand{\nbar}{\ov{n}}
\newcommand{\nnbar}{n\ov{n}}
\newcommand{\muU}{\mu_\text{U}}


\newcommand{\cpn}{CP$^{N-1}$\,}
\newcommand{\CPC}{CP($N-1$)$\times$C }
\newcommand{\CPCn}{CP($N-1$)$\times$C}


\newcommand{\lar}{\lambda_R}
\newcommand{\lal}{\lambda_L}
\newcommand{\larl}{\lambda_{R,L}}
\newcommand{\lalr}{\lambda_{L,R}}
\newcommand{\blar}{\ov{\lambda}{}_R}
\newcommand{\blal}{\ov{\lambda}{}_L}
\newcommand{\blarl}{\ov{\lambda}{}_{R,L}}
\newcommand{\blalr}{\ov{\lambda}{}_{L,R}}


\newcommand{\bgamma}{\ov{\gamma}}
\newcommand{\bpsi}{\ov{\psi}{}}
\newcommand{\bphi}{\ov{\phi}{}}
\newcommand{\bxi}{\ov{\xi}{}}


\newcommand{\qt}{\wt{q}}
\newcommand{\bq}{\ov{q}}
\newcommand{\bqt}{\overline{\widetilde{q}}}


\newcommand{\eer}{\epsilon_R}
\newcommand{\eel}{\epsilon_L}
\newcommand{\eerl}{\epsilon_{R,L}}
\newcommand{\eelr}{\epsilon_{L,R}}
\newcommand{\beer}{\ov{\epsilon}{}_R}
\newcommand{\beel}{\ov{\epsilon}{}_L}
\newcommand{\beerl}{\ov{\epsilon}{}_{R,L}}
\newcommand{\beelr}{\ov{\epsilon}{}_{L,R}}


\newcommand{\bi}{{\bar \imath}}
\newcommand{\bj}{{\bar \jmath}}
\newcommand{\bk}{{\bar k}}
\newcommand{\bl}{{\bar l}}
\newcommand{\bmm}{{\bar m}}


\newcommand{\nz}{{n^{(0)}}}
\newcommand{\no}{{n^{(1)}}}
\newcommand{\bnz}{{\ov{n}{}^{(0)}}}
\newcommand{\bno}{{\ov{n}{}^{(1)}}}
\newcommand{\Dz}{{D^{(0)}}}
\newcommand{\Do}{{D^{(1)}}}
\newcommand{\bDz}{{\ov{D}{}^{(0)}}}
\newcommand{\bDo}{{\ov{D}{}^{(1)}}}
\newcommand{\sigz}{{\sigma^{(0)}}}
\newcommand{\sigo}{{\sigma^{(1)}}}
\newcommand{\bsigz}{{\ov{\sigma}{}^{(0)}}}
\newcommand{\bsigo}{{\ov{\sigma}{}^{(1)}}}


\newcommand{\rrenz}{{r_\text{ren}^{(0)}}}
\newcommand{\bren}{{\beta_\text{ren}}}


\newcommand{\Tr}{\text{Tr}}
\newcommand{\Ts}{\text{Ts}}
\newcommand{\dm}{\hat{{\scriptstyle \Delta} m}}
\newcommand{\dmdag}{\hat{{\scriptstyle \Delta} m}{}^\dag}
\newcommand{\mhat}{\widehat{m}}
\newcommand{\deltam}{{\scriptstyle \Delta} m}
\newcommand{\nvac}{\vec{n}{}_\text{vac}}


\newcommand{\ie}{{\it i.e.}~}
\newcommand{\eg}{{\it e.g.}~}
\newcommand{\ansatz}{{\it ansatz} }




\begin{document}




%%%%%%%%%%%%%%%%%%%%%%%%%%%%%%%%%%%%%%%%%%%%%%%%%%%%%%%%%%%%%%%%%%%%%%%%%%%%%%%%
%                                                                              %
%                                                                              %
%                            T I T L E   P A G E                               %
%                                                                              %
%                                                                              %
%%%%%%%%%%%%%%%%%%%%%%%%%%%%%%%%%%%%%%%%%%%%%%%%%%%%%%%%%%%%%%%%%%%%%%%%%%%%%%%%
\begin{titlepage}


\begin{center}
{  \Large \bf  A Model of Non-Abelian Magnetohydrodynamics}
\end{center}

\vspace{2mm}


\end{titlepage}




%%%%%%%%%%%%%%%%%%%%%%%%%%%%%%%%%%%%%%%%%%%%%%%%%%%%%%%%%%%%%%%%%%%%%%%%%%%%%%%%
%                                                                              %
%                                                                              %
%                            I N T R O D U C T I O N                           %
%                                                                              %
%                                                                              %
%%%%%%%%%%%%%%%%%%%%%%%%%%%%%%%%%%%%%%%%%%%%%%%%%%%%%%%%%%%%%%%%%%%%%%%%%%%%%%%%
\section{Introduction}
\setcounter{equation}{0}

	Consider a Lagrangian which would roughly form a generalization of the Clebsch parametrization,
\beq
\label{lagrangian}
	\ml    ~~=~~    2\, \Tr\, J_\mu\,  \md^{\text{A}\mu} \theta   ~~-~~  f(n)  ~~+~~
			\frac{1}{2}\, \Tr\, G_{\mu\nu}^2\,.
\eeq
	Here,
\beq
	J_\mu    ~~=~~    J_\mu^a\, T^a
\eeq
	is the non-Abelian current,
\beq
	\md^\text{A}_\mu    ~~=~~    \p_\mu  ~~+~~  [ A_\mu\,, \cdot\,\, ]
\eeq
	is the gauge derivative in the adjoint representation,
\beq
	\theta    ~~=~~    \theta^a\, T^a
\eeq
	is the analogue of the Clebsch potential,
\beq
	f(n)\,, \qquad\qquad n ~=~ \sqrt{ J_\mu^a\, J^{\mu a} }
\eeq
	is a function determining the dynamics of the fluid (interactions and so on),
	and
\beq
	G_{\mu\nu}    ~~=~~    \p_{[\mu} A_{\nu]}  ~~+~~ [ A_\mu\, A_\nu ]
\eeq
	is the gluon field strength.
%%	The derivative in square brackets $ [\, \md^\mu\, \underline{~~}\, ] $ is just a derivative
%%	acting in the adjoint representation.

	Unlike QED, where the current is gauge invariant, in a non-Abelian theory, both $ J_\mu $
	and $ \theta $ must transform as adjoints of SU($N$).

	The equations of motion for Eq.~\eqref{lagrangian} are
\begin{align}
%
\label{cconserv}
	\md^\text{A}_\mu\, J^\mu     & ~~=~~    0\,,
	\\[3.4mm]
%
\label{maxwell}
	\md^\text{A}_\mu\, G^{\mu\nu}     & ~~=~~    [\, \theta\, J^\nu \,]\,,
	\\[2mm]
%
\label{flow}
	\md^\text{A}_\mu\, \theta     & ~~=~~    \frac{1}{n}\, f^\prime(n)\, J_\mu\,.
\end{align}
	The first equation is just the covariant conservation of the current $ J^\mu $.
	The second equation --- the Maxwell equation --- immediately implies the conservation
	of another current
\beq
\label{aconserv}
	\md^\text{A}_\mu\, \big[\, \theta\, J^\mu \,\big]    ~~=~~    0\,.
\eeq
	Although one can use the continuity equation \eqref{cconserv} here, it does not mean
	that $ \md^\text{A}_\mu\, \theta  \,=\, 0 $.
	Instead, as Eq.~\eqref{flow} shows, this gradient is proportional to the current,
        and so commutes with it.
	The current in \eqref{aconserv} is the one that the gluons actually see.
	Although it differs from the original one in \eqref{lagrangian},
	by a non-Abelian rotation $ \theta $,
	it is the one which affects the colour gauge field.


	Finally, equation \eqref{flow} determines the dynamics (flow) of the fluid.
	Let us first plug this equation into the original Lagrangian \eqref{lagrangian}, to find
\beq
	\ml    ~~=~~    n\, f^\prime(n)  ~-~  f(n)  ~+~ \frac{1}{2}\, \Tr\, G_{\mu\nu}^2\,.
\eeq
	This roughly corresponds to our vision of the starting point for the ``integrated'' action for
	hydrodynamics.


%%	Now let us take the curl of Eq.~\eqref{flow}, and contract it with $ J^\mu $,
%%\beq
%%	\big[\, J^\mu,~ \p_{[\mu} \Big( \frac{f^\prime}{n}\, J_{\nu]} \Big) \,]  ~+~
%%	\frac{f^\prime}{n}\, \big[\, J^\mu\, [\, A_{[\mu}\, J_{\nu]} \,] \,\big]    ~~=~~
%%	\big[\, J^\mu\, [\, G_{\mu\nu}\, \theta \,] \,\big] \,.
%%\eeq
%%	The left hand side should determine the Euler equation for the fluid, with the right hand side
%%	being interpreted as the non-Abelian Lorentz force acting on the current.
%%	Although an unimportant detail, by appearance this is the Lorentz force
%%	induced by either of the currents
%%	$ J^\mu $ and $ [\, \theta\, J^\mu \,] $,
%%\beq
%%	\big[\, J^\mu\, [\, G_{\mu\nu}\, \theta \,] \,\big]    ~~=~~
%%	-\, \big[\, G_{\mu\nu}\, [\, \theta\, J^\mu \,] \,\big]  ~-~
%%	    \big[\, \theta\, [\, G_{\mu\nu}\, J^\mu \,] \,\big]\,.
%%\eeq

	We now can calculate the (covariant) curl of Eq.~\eqref{flow}, to obtain
\beq
\label{curl}
	\partial_{[\mu}\, \Big( \frac{f^\prime}{n}\, J_{\nu]} \Big)  ~~+~~
	\frac{f^\prime}{n}\, \big[\, A_{[\mu}\, J_{\nu]} \,\big]    ~~=~~
	[\, G_{\mu\nu}\, \theta \,]\,.
\eeq
	As customary, we split the currents $ J_\mu^a $ into their ``lengths'' --- $ n^a $,
	and the directions --- $ u_\mu^a $,
\beq
	J_\mu    ~~=~~    n^a\, u_{\mu}^a\, T^a\,.
\eeq
	Let us contract equation \eqref{curl} with $ J_\mu^a T^a $ for a certain $ a $,
	and multiply it by $ u_\mu^a $.
	To emphasize that there is \emph{no summation} in index $ a $, we denote it as $ \slashed{a} $,
\beq
\label{uchange}
	u_\mu^\slashed a\, \p_{[\mu} \Big( \frac{f^\prime}{n}\, J_{\nu]}^\slashed a \Big)  ~~+~~
	\frac{f^\prime}{n}\, f^{\slashed a b c}\, u_\mu^\slashed a\, A_{[\mu}^b\, J_{\nu]}^c    ~~=~~
	f^{\slashed a b c}\, u_\mu^\slashed a\, G_{\mu\nu}^b\, \theta^c\,.
\eeq
	The first term is supposed to give the change of the current $ u_\mu^a $,
	the second term is non-Abelian in nature, while the third term has the appearance of the Lorentz force.

	To check, we find the non-relativistic limit of Eq.~\eqref{uchange}, $ u^\mu ~\approx~ (\,1,~ \vec{v}\,) $,
\beq
	\p_0\, v_k^\slashed a  ~+~  (\vec{v}{}^\slashed{a}\, \vec{\p})\, v_k^\slashed a    \,~=~\,
	-\, \p_k\, L^\slashed{a}  ~-~  \frac{1}{(n^\slashed{a})^2}\, f^{\slashed a b c}\, J_\mu^\slashed{a}\, J_{[\mu}^b\, A_{k]}^c  ~+~
	\frac{1}{f^\prime / n\, (n^{\slashed a})^2}\, f^{\slashed a b c}\, J_\mu^\slashed{a}\, G_{\mu k}^b\, \theta^c\,,
\eeq
	where
\beq
	L^\slashed{a}    ~~=~~    \ln\, \Big( \frac{f^\prime}{n}\, n^\slashed{a} \Big)
\eeq
	gives the force resulting from the internal dynamics.
	The last term indeed has the form of the Lorentz force acting on the current $ [\, \theta\, J^\mu \,] $.


%%%%%%%%%%%%%%%%%%%%%%%%%%%%%%%%%%%%%%%%%%%%%%%%%%%%%%%%%%%%%%%%%%%%%%%%%%%%%%%%
%                                                                              %
%                  E N E R G Y   M O M E N T U M   T E N S O R                 %
%                                                                              %
%%%%%%%%%%%%%%%%%%%%%%%%%%%%%%%%%%%%%%%%%%%%%%%%%%%%%%%%%%%%%%%%%%%%%%%%%%%%%%%%
\subsection{Energy-Momentum Tensor}
	Let us calculate the energy-momentum tensor of the Chromohydrodynamic (CHD) part of Lagrangian \eqref{lagrangian}.
	In a gauge theory it is easier to vary with respect to metric $ g^{\mu\nu} $ rather than obtaining
	a symmetric canonical energy-momentum tensor.

	One has to also take into account that $ f(n) $ also depends on the metric via $ n $.
	We find,
\beq
\label{emtensor}
	T_{\mu\nu}^\text{CHD}    ~~=~~    \frac{(n^a)^2}{n^2}\, n f^\prime \cdot u_\mu^a u_\nu^a  ~~-~~
					  g^{\mu\nu}\, \big(\, n f^\prime ~-~ f(n) \,\big)\,.
\eeq
	We observe that we have a single common pressure
\beq
	p    ~~=~~    n f^\prime(n) ~-~ f(n)
\eeq
	and $ N^2 - 1 $ energy densities
\beq
	\rho^a    ~~=~~    \lgr \Big( \frac{n^a}{n} \Big)^2 ~-~ 1 \rgr n f^\prime  ~~+~~ f(n)
\eeq
	which sum up to $ f(n) $.

        We can go further than Eq.~\eqref{lagrangian} and declare that the Lagrangian
        is a generic function of $ n $ and of the non-Abelian field strength $ \varphi^2 $,
\beq
\label{generic}
        \ml    ~~=~~    2\, \Tr\, J_\mu\,  \md^{\text{A}\mu} \theta   ~~-~~  f(n, \varphi^2)\,,
\eeq
        where
\beq
        \varphi^2    ~~=~~    \Tr\, G_{\mu\nu}^2\,.
\eeq
	The expression in Eq.~\eqref{emtensor} still gives the contribution of the current to the energy-momentum tensor,
	while the gluonic contribution to the latter is
\beq
\label{T-gluon}
	T_{\mu\nu}^\text{gluon}    ~~=~~    -\, 4\, f^\prime_{\varphi^2} \cdot \Tr\, G^{\mu\lambda}\, G^{\nu\lambda}
				   ~~+~~    g_{\mu\nu}\, f(n, \varphi^2)\,.
\eeq
	We understand here that the last term in this equation is shared with Eq.~\eqref{emtensor}.




%%%%%%%%%%%%%%%%%%%%%%%%%%%%%%%%%%%%%%%%%%%%%%%%%%%%%%%%%%%%%%%%%%%%%%%%%%%%%%%%
%                                                                              %
%                                                                              %
%                        S C A L A R   V A R I A B L E S                       %
%                                                                              %
%                                                                              %
%%%%%%%%%%%%%%%%%%%%%%%%%%%%%%%%%%%%%%%%%%%%%%%%%%%%%%%%%%%%%%%%%%%%%%%%%%%%%%%%
\section{Scalar variables}

	Now let us express the gauge field and the current in terms of scalar variables $ X^a $, $ Y^a $, $ Z^a $.
\beq
\label{defA}
        A_\mu    ~~=~~    X^a\, \p_\mu Y^a\, \cdot\, T^a\,,
\eeq
	and, correspondingly, the field strength is
\beq
\label{defG}
	G_{\mu\nu}    ~~=~~    \md_{[\mu}\, \md_{\nu]}\,.
\eeq
	In terms of the scalars, the field strength is
\beq
\label{Ga}
	G{}_{\mu\nu}^a    ~~=~~    \p_{[\mu} X^a\, \p_{\nu]} Y^a  ~~+~~
				   f^{abc}\, X^b\, X^c\, \p_\mu Y^b\, \p_\nu Y^c\,.
\eeq
        Notice we have not introduced the coupling constant $ g $ yet, for simplicity of calculations.
	When needed, it is introduced simply by rescaling $ A_\mu ~\to~ g\, A_\mu $.
	The field strength will have to be divided by one power of $ g $ at the same time.
	But that will be done later.

        We do not treat $ X $ and $ Y $ as matrices separately, instead their combination \eqref{defA} is a matrix.
	In order to construct an explicitly conserved current, we introduce $ N^2 - 1 $ variables $ Z^a $,
	which now we treat as a matrix
\beq
	Z    ~~=~~    Z^a\, T^a\,.
\eeq
	We define,
\beq
\label{defJ}
	J^\mu    ~~=~~    \epsilon^{\mu\nu\rho\sigma}\, [\, G_{\nu\rho},\, \p_\sigma Z\, ]\,.
\eeq
	This current is covariantly conserved identically,
\beq
\label{cont}
	\md^{A}_\mu\, J^\mu    ~~\equiv~~    0\,.
\eeq
	This can be seen as follows.
	The field strength $ G_{\mu\nu} $ acting as a commutator on a matrix $ M $ can be written as
	a sequence of two gauge derivatives
\beq
	[\, G_{\mu\nu}\,\, M \,]    ~~=~~    \md^\text{A}_{[\mu}\, \md^\text{A}_{\nu]}\, M\,,
\eeq
	quite similar to the definition \eqref{defG}.
	Following this, equation \eqref{defJ} can be rewritten as
\beq
	J^\mu    ~~=~~    2\, \epsilon^{\mu\nu\rho\sigma}\, \md^\text{A}_\nu\, \md^\text{A}_\rho\, \p_\sigma Z\,.
\eeq
	Then,
\beq
	\md^\text{A}_\mu\, J^\mu    ~~=~~    2\, \epsilon^{\mu\nu\rho\sigma}\,
					     \md^\text{A}_\mu\, \md^\text{A}_\nu\, \md^\text{A}_\rho\, \p_\sigma Z\,
				    ~~=~~    0\,,
\eeq
	due to the Bianchi identity for $ \md^\text{A}_\mu $,
\beq
	\epsilon^{\mu\nu\rho\sigma}\, \md^\text{A}_\nu\, \md^\text{A}_\rho\, \md^\text{A}_\sigma    ~~=~~    0\,.
\eeq
	Precisely because of the Bianchi identity we could not have used a full gauge derivative $ \md_\sigma Z $
	in the definition of $ J^\mu $ \eqref{defJ}, or the current would identically have been zero.

	Let us analyze equations \eqref{cconserv}--\eqref{flow} now in terms of the number of degrees of freedom.
	The first equation \eqref{cconserv} is now satisfied identically.
	The third equation \eqref{flow} is in fact the definition of the Clebsch potential $ \theta $.
	In general, there are not enough degrees of freedom in $ \theta $ to represent an arbitrary
	(conserved) current $ J^\mu $ in the right hand side.
	We can assume that the currents are not highly circular, and so admit a parametrization in terms of
	just one Clebsch parameter.
	Alternatively, we could have introduced more Clebsch parameters --- matrices $ \alpha $ and $ \beta $.
	This would just make the right hand side of equation \eqref{maxwell} more complicated, as well as that of \eqref{curl}.
	No conceptual difference would have arisen.
	Finally, equation \eqref{maxwell} is the non-Abelian Maxwell equation.
	There are $ 4 \times (N^2 - 1 ) $ equations on $ 3 \times (N^2 - 1) $ variables
	$ X^a $, $ Y^a $ and $ Z^a $
	($ \theta $ does not count as it is expressed in terms of $ X^a $, $ Y^a $ and $ Z^a $).
	Such a situation is not overdefined, as from equation \eqref{maxwell} immediately follows
	the continuity equation \eqref{cont}, which has been satisfied explicitly.
	Subtracting the $ N^2 - 1 $ degrees of the latter equation, we have precisely
	$ 3 \times (N^2 - 1) $ equations for $ 3 \times (N^2 - 1) $ variables.


	Note, that because of the commutator in \eqref{defJ}, we do \emph{not} have the decoupling
	property of the currents in the weak coupling limit --- the presence of the structure constant $ f^{abc} $
	in that commutator would mix all the currents.
	The field strength $ G_{\mu\nu}^a $ in \eqref{Ga} in the limit $ g ~\to~ 0 $
	(which corresponds to dropping the second term), obviously decomposes into a set of $ N^2 - 1 $
	fields.
	However, it is unclear whether the decomposition of the ``effective'' current should take place,
	as even in the case of SU(2), in the weak-coupling limit, all quarks and gluons are still mixed
	due to the non-Abelian nature of the interactions.
	Indeed, two of the three gluons mix both quarks, while all the gluons are mixed due to the triple-gluon
        vertex.
	If one takes the $ g ~=~ 0 $ limit to completely suppress such a vertex, the usual quark-gluon
        $ J^\mu A_\mu $ vertex of the QCD will also disappear, in which case there would be no sense of talking about a plasma.


%%%%%%%%%%%%%%%%%%%%%%%%%%%%%%%%%%%%%%%%%%%%%%%%%%%%%%%%%%%%%%%%%%%%%%%%%%%%%%%%
%                                                                              %
%       T R E E - L E V E L   E N E R G Y   M O M E N T U M   T E N S O R      %
%                                                                              %
%%%%%%%%%%%%%%%%%%%%%%%%%%%%%%%%%%%%%%%%%%%%%%%%%%%%%%%%%%%%%%%%%%%%%%%%%%%%%%%%
\subsection{Tree-level energy momentum tensor}
	In the Abelian plasma, the field strength $ F^{\mu\nu} $ happens to be a bivector,
	which helps a lot in understanding the structure of the energy-momentum tensor.
	In our case, the field strength \eqref{Ga} obviously is not a bivector, but
	if we make a perturbative expansion, we can treat it as such.
	
	In order to make a perturbative expansion, one normally redefines $ A^\mu ~\to~ g A^\mu $
	while at the same time $ G^{\mu\nu} ~\to~ g^{-1} G^{\mu\nu} $.
	We prefer not to do this explicitly, while keeping in mind that each additional power of $ X $ or $ Y $
	carries an extra power of the coupling constant.

	Then the "tree-level" field strength happens to be a bivector,
\beq
	G_{\mu\nu}^{(0)a}    ~~=~~    \p_{[\mu} X^a\, \p_{\nu]} Y^a\,,
\eeq
	and can be treated the same way as in the Abelian plasma.
	Explicitly, the dual field strength is defined to be
\beq
	\wt{G}{}_{\mu\nu}^{(0)a}    ~~=~~    \phi^\slashed{a}\, w^{\slashed{a}[\rho}\, u^{\sigma]\slashed{a}},
\eeq
	where
\beq
	w^{\slashed{a}\mu}\, w^{\slashed{a}}_\mu    ~~=~~    \mp\,1\,
\eeq
	(the upper sign is for Euclidean and the lower sign for Minkowski spaces correspondingly),
	and
\beq
	( \varphi^\slashed{a} )^2    ~~=~~    \frac{1}{2}\, \big( \p_{[\mu} X^\slashed{a}\, \p_{\nu]} Y^\slashed{a} \big)^2\,.
\eeq

	With these definitions the field strength \eqref{Ga} can be re-written as
\beq
\label{G-pert}
	G_{\mu\nu}^\slashed a    ~~=~~    \epsilon_{\mu\nu\rho\sigma}\, \phi^\slashed{a} \cdot w^{a\rho}\, u^{a\sigma}
				 ~~+~~  ( f^{abc}\, A^b\, A^c )\, q_\mu^b\, q_\nu^c\,.
\eeq
	We have also introduced here a unit vector $ q^a_\mu $ directed along potential $ A_\mu^a $,
\beq
	A_\mu^a    ~~=~~    X^\slashed a\, \p_\mu Y^\slashed a    ~~\equiv~~    A^\slashed a \cdot q_\mu^\slashed a\,,
	\qquad\qquad
	( q_\mu^\slashed a )^2    ~~=~~    1\,.
\eeq
	The expansion in the coupling constant $ g $ will correspond to an expansion in the (even) powers of $ q_\mu^a $.

	Substituting expansion \eqref{G-pert} into Eq.~\eqref{T-gluon} we find
\begin{align}
%
\notag
	T_\text{gluon}^{\mu\nu}    & ~~=~~    \frac{f^\prime_\varphi}{\varphi}\, (\varphi^a)^2\,
					      \lgr \pm\, w^{a\mu}\, w^{a\nu}  ~+~  u^{a\mu}\, u^{a\nu} \rgr    ~~-~~
	\\[2mm]
%
\label{full-tensor}
				   & ~~-~~  \frac{f^\prime_\varphi}{\varphi}\, \varphi^a\,
					    Q^{abc}_{\underline{\mu}\lambda}\, q^b_{\underline{\nu}}\, q^c_\lambda
				     ~~-~~  \frac{f^\prime_\varphi}{\varphi}\,
					    R^{abcd}\, ( q^a_\kappa\, q^c_\kappa )\, q^b_\mu\, q^d_\nu    ~~-~~
	\\[2mm]
%
\notag
				   & ~~-~~  g^{\mu\nu}
					    \lgr \frac{f^\prime_\varphi}{\varphi}\, (\varphi^a)^2  ~-~  f \rgr .
\end{align}
	Here the underlined subscripts $ \underline{\mu} $ and $ \underline{\nu} $ assume symmetrization
	in these indices.
	The coefficients $ Q^{abc}_{\mu\nu} $ and $ R^{abcd} $ are defined as
\begin{align*}
%
	Q^{abc}_{\mu\nu}    & ~~=~~    \epsilon_{\mu\nu\rho\sigma}\,
				       f^{\slashed a\slashed b\slashed c}\, A^\slashed{b}\, A^\slashed{c}\,
				       w^{a\rho}\, u^{a\sigma}\,,
	\\[2mm]
%
	R^{abcd}    & ~~=~~    f^{f\slashed a\slashed b}\, A^\slashed{a}\, A^\slashed{b} \cdot
			       f^{f\slashed c\slashed d}\, A^\slashed{c}\, A^\slashed{d}\,.
\end{align*}
	Tensor \eqref{full-tensor} is a complete expression for the energy-momentum tensor,
	even though it does have the appearance of an expansion in powers of $ q^a_\mu $.

	To obtain a true expansion in powers of the coupling constant, we observe that the flux $ \varphi $
	needs to be expanded beforehand,
\beq
	\varphi^2    ~~=~~    (\varphi^a)^2  ~~+~~ \varphi^a\, Q^{abc}_{\kappa\lambda}\, q^b_\kappa\, q^c_\lambda
		     ~~+~~ \frac 1 2\, R^{abcd}\, (q^a_\kappa\, q^c_\kappa)\, (q^b_\lambda\, q^d_\lambda)\,.
\eeq
	Substituting this, we obtain the leading order contribution to the gluon energy-momentum tensor,
\begin{align}
%
\notag
	T_\text{gluon}^{\mu\nu}    & ~~=~~    2\, f^\prime_{\varphi^2} \bigg|_{\varphi_{(0)}}\,
					      (\varphi^a)^2 \lgr \pm\, w^{a\mu}\, w^{a\nu}  ~+~  u^{a\mu}\, u^{a\nu} \rgr
				     ~~-~~
	\\[1mm]
%
\label{tree-tensor}
				   & ~~-~~    g^{\mu\nu} \lgr 2\, f^\prime_{\varphi^2}\, \varphi^2 ~-~ f \rgr \bigg|_{\varphi_{(0)}}
				     ~~+~~    O\big(\, q^2 \,\big)\,.
\end{align}
	Here we have denoted the "tree-level" $ \varphi $ as $ \varphi_{(0)} $\,
\beq
	\varphi_{(0)}    ~~=~~    \sqrt{ (\varphi^a)^2 }\,.
\eeq
	We can imagine that this hydrodynamic limit is reached in the regime of a very weak coupling.
	The corrections to \eqref{tree-tensor} are easily found
\begin{align}
%
\notag
	\Delta T_\text{gluon}^{\mu\nu}    & ~~=~~    2\, f''_{\varphi^2}\bigg|_{\varphi_{(0)}} \cdot
						     \varphi^b\, Q^{bcd}_{\kappa\lambda}\, q^c_\kappa q^d_\lambda \cdot
						     (\varphi^a)^2
						     \lgr \pm\, w^{a\mu}\, w^{a\nu} ~+~ u^{a\mu}\, u^{a\nu} ~-~ g^{\mu\nu} \rgr
					    ~~-~~
	\\[1mm]
%
\label{tensor-corr}
					  & ~~-~~    2\, f''_{\varphi^2}\bigg|_{\varphi_{(0)}} \cdot
						     \varphi^a\, Q^{abc}_{\underline{\mu}\lambda}\, q^b_{\underline \nu}\, q^c_\lambda
					    ~~+~~ O\big(\, q^4 \,\big)\,.
\end{align}
	Again, here the upper sign of $ \pm $ corresponds to Euclidean space, and the lower sign to Minkowski space.




%%%%%%%%%%%%%%%%%%%%%%%%%%%%%%%%%%%%%%%%%%%%%%%%%%%%%%%%%%%%%%%%%%%%%%%%%%%%%%%%
%                                                                              %
%                                                                              %
%                 S I N G L E   C U R R E N T   A P P R O A C H                %
%                                                                              %
%                                                                              %
%%%%%%%%%%%%%%%%%%%%%%%%%%%%%%%%%%%%%%%%%%%%%%%%%%%%%%%%%%%%%%%%%%%%%%%%%%%%%%%%
\section{Single current approach}
	Following Jackiw \cite{}, we can construct a scalar conserved current $ j^\mu $.
	In the latter approach, all currents flow in the same direction. 
	Using our variables $ X^a $, $ Y^a $ and $ Z^a $, we introduce,
\beq
	j^\mu    ~~=~~    \epsilon^{\mu\nu\rho\sigma}\, f_{abc}\, \p_\nu X^a\, \p_\rho Y^b\, \p_\sigma Z^c\,.
\eeq
	The Eckart's form \cite{} implies\,
\beq
	J^\mu    ~~=~~    Q \cdot j^\mu\,.
\eeq
	With these definitions, Jackiw's approach may be used.
	However, besides variables $ X^a $, $ Y^a $ and $ Z^a $, we also have to introduce $ Q^a $.
	There are now too many degrees of freedom for the Maxwell's equations --- we earlier
	argued that just $ X^a $, $ Y^a $ and $ Z^a $ are enough.
	Nevertheless, Jackiw's results can be used unchanged.




\vspace{1cm}
	Finally, we note that neither of the two approaches that we have presented exhibit the effect
	of freezing of the chromomagnetic lines.
	Jackiw was able to derive the hydrodynamic equations (and so were we), without demanding
	this freezing.
	



%%%%%%%%%%%%%%%%%%%%%%%%%%%%%%%%%%%%%%%%%%%%%%%%%%%%%%%%%%%%%%%%%%%%%%%%%%%%%%%%
%                                                                              %
%                                                                              %
%                             C O N C L U S I O N S                            %
%                                                                              %
%                                                                              %
%%%%%%%%%%%%%%%%%%%%%%%%%%%%%%%%%%%%%%%%%%%%%%%%%%%%%%%%%%%%%%%%%%%%%%%%%%%%%%%%
\section{Conclusions}




%%%%%%%%%%%%%%%%%%%%%%%%%%%%%%%%%%%%%%%%%%%%%%%%%%%%%%%%%%%%%%%%%%%%%%%%%%%%%%%%
%                                                                              %
%                                                                              %
%                          A C K N O W L E D G M E N T S                       %
%                                                                              %
%                                                                              %
%%%%%%%%%%%%%%%%%%%%%%%%%%%%%%%%%%%%%%%%%%%%%%%%%%%%%%%%%%%%%%%%%%%%%%%%%%%%%%%%
\section*{Acknowledgments}

\small
\begin{thebibliography}{99}
\itemsep -2pt

\end{thebibliography}


\end{document}
